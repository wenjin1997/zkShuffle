% \documentclass[11pt,twoside,a4paper]{article}
% \usepackage{times}

% \usepackage{xeCJK}

% \setmainfont{Times New Roman}

% \setCJKmainfont{Songti SC}
\documentclass[10pt]{ctexart}
% \usepackage[UTF-8]{ctex}
\usepackage{amsmath}
\usepackage{amsthm} % 根据 amsthm 的手册, amsthm 的加载要在 amsmath 之后
\usepackage{amssymb}  %为了能使用\mathbb{H} 
\usepackage{booktabs}
\usepackage{multirow}
\usepackage{tabularx}
\usepackage{xcolor}
\usepackage[colorlinks,linkcolor=blue]{hyperref} % 使用超链接
\usepackage{pdfpages}
\usepackage{geometry}
\geometry{a4paper,scale=0.8}
\usepackage{graphicx} %插入图片的宏包
\usepackage{float} %设置图片浮动位置的宏包
\usepackage{subfigure} %插入多图时用子图显示的宏包
\usepackage{graphicx}

\usepackage{listings}

\lstset{
 columns=fixed,       
 numbers=left,                                        % 在左侧显示行号
 numberstyle=\tiny\color{gray},                       % 设定行号格式
 frame=none,                                          % 不显示背景边框
 backgroundcolor=\color[RGB]{245,245,244},            % 设定背景颜色
 keywordstyle=\color[RGB]{40,40,255},                 % 设定关键字颜色
 numberstyle=\footnotesize\color{darkgray},           
 commentstyle=\it\color[RGB]{0,96,96},                % 设置代码注释的格式
 stringstyle=\rmfamily\slshape\color[RGB]{128,0,0},   % 设置字符串格式
 showstringspaces=false,                              % 不显示字符串中的空格
%language=c++,                                        % 设置语言
}

\newtheorem{definition}{定义}
\newtheorem{lemma}{引理}
\newtheorem{theorem}{定理}



\title{JubJub曲线}
\author{谢文进}
\date{\today}
\begin{document}
\maketitle
\tableofcontents

\section{JubJub曲线}
\subsection{Baby JubJub vs. JubJub}
\begin{table}[h]
  \centering
  \setlength{\tabcolsep}{4mm}
  \caption{Baby JubJub vs. JubJub} \label{tab:example1}
  \begin{tabular}{ccc}
  \toprule
    & Baby JubJub & JubJub\\
  \midrule
  Montgomery curve & $y^2 = x^3 + 168698x^2 + x$ & $y^2 = x^3 + 40962x^2 + x$ \\
  Twisted Edwards curve & $168700x^2 + y^2 = 1 + 168696x^2y^2$ & $40964x^2 + y^2 = 1 + 40960x^2y^2$ \\
  After scaling & $-x^2+y^2 = 1 + (-168696/168700)x^2y^2$ & $-x^2+y^2 = 1 + (-10240/10241)x^2y^2$\\
  \bottomrule
\end{tabular}
\end{table}

\subsection{Montgomery curve}
BLS12381的阶:

$p =52435875175126190479447740508185965837690552500527637822603658699938581184513$

JubJub曲线:$y^2=x^3+40962x^2+x$

$n = h \times l$,其中$h = 8$
$$
n = 52435875175126190479447740508185965837647370126978538250922873299137466033592
$$
$$
l = 6554484396890773809930967563523245729705921265872317281365359162392183254199
$$

生成元:$G_0^M=(x_0^M,y_0^M)$

$$
x_0^M = 10,
$$
$$
y_0^M=4864016555691628132658688815665199693566189978262460729570611510575653263443
$$

base point $G_1^M = (x_1^M,y_1^M)$
$$
x_1^M = 18662558417428907826588413824421338451096304318249267443782476967941530951665,
$$
$$
y_1^M=10829583678449754674258847202008243338777477689250089851084221267577753407818
$$
% \subsubsection{BabyJubJub curve}
% BN254的素数阶:

% $p =21888242871839275222246405745257275088548364400416034343698204186575808495617$

% JubJub曲线:$y^2=x^3+16898x^2+x$

% 生成元:$G_0^M=(x_0^M,y_0^M)$

% $$
% x_0^M = 7,
% $$
% $$
% y_0^M=4864016555691628132658688815665199693566189978262460729570611510575653263443
% $$

% base point $G_1^M = (x_1^M,y_1^M)$
% $$
% x_1^M = 18662558417428907826588413824421338451096304318249267443782476967941530951665,
% $$
% $$
% y_1^M=10829583678449754674258847202008243338777477689250089851084221267577753407818
% $$
\subsection{Twisted Edwards curve}
曲线方程:$40964x^2 + y^2 = 1 + 40960x^2y^2$

生成元:$G_0=(x_0,y_0)$

$$
x_0 = 50210964013865202807697778192253911217867412634541528443086291916476367487291 ,
$$
$$
y_0=9533795486386580087172316456033811970489191363732297785927937945443378397185
$$

base point $G_1 = (x_1,y_1)$
$$
x_1 = 25048732578063176751608348748134148938511950496662102134944680124345451629928 ,
$$
$$
y_1=37193546711722353718211339597021920218737743307281823076963067517006020382455
$$
\subsection{scale}
Theorem 6. Consider a twisted Edwards curve defined over Fp given by equation $ax^2 + y^2 = 1 + dx^2y^2$. If $−a$ is a square in $\mathbb{F}_p$, then the map $(x, y) \rightarrow (x/\sqrt{-a}, y)$ defines the curve $−x^2 + y^2 = 1 + (−d/a)x^2y^2$. We denote by $f = \sqrt{-a}$ the scaling factor.

因此曲线$40964x^2 + y^2 = 1 + 40960x^2y^2$等价于曲线$-x^2+y^2 = 1 + (-10240/10241)x^2y^2$. scaling factor为$f = \sqrt{-40964}$.
\end{document}